Puntos a  tener en cuenta para el proyecto de salesdatepredictions   ---


*********Tener en cuenta :   ************************************************************************************
Se adjunta archivo de word en el cual se indica como se ejecuta el proyecto con capturas de pantalla funcionales.




cambiar la conexion en appsetings --------------------------------------*********************----********


 "ConnectionStrings": {
     "MiConexion": "Server=FERCHO\\SQLEXPRESS;Database=StoreSample;Trusted_Connection=True;"
 },
 
 cambiarla a la que se tenga configurada en sql server ----------------**************-----********
 
 se crea el proyecto SalesDatePredictionTest para implementar pruebas unitarias xunit----------------************--------------------
 
 
 ---********** ---IMPORTANTE*****************-----------------------------*******************************
 +++++++++++++++++++++++++++++++++++++++++++++++++++++++++++++++++++++++++++++++++++++++++++++++++++++
 
 
SalesDatePredictionTest ----------------------------------------************************


dotnet new xunit -o TuProyectoDePruebas  --- comando para crear el proyecto de pruebas 
dotnet add package xunit  --- agregar paquete xunit

 

 ++++++++---------------------------------------*************************************************
 
 
 tener presente las URLS son muy importantes para poderunificar el back con el front para ello :  
 *****Se indica las rutas donde se encuentran las URLS correspondientes.*********
 
 URL API CustomerOrderPredictions  ------**********************
 https://localhost:7170/CustomerOrderPredictions  
 
 se encuentra configurada en 
 "sales-date-prediction.service.ts" cualquier cambio de URL para la api se debe realizar aqui
 
 --------------------------------------------------------------*****************
 
 URL API Orders--------***************************
 https://localhost:7170/Orders
 
 se encuentra configurada en "src/app/services/order.service.ts"  cualquier cambio de URL para la api se debe realizar aqui
 
 -----------------------------------------------------------------******************************************
 
 URL API https://localhost:7170/OrdersCreate -------------------*************************
 
 se encuentra configurada en  "src/app/services/order-create.service.ts" ualquier cambio de URL para la api se debe realizar aqui
 
 -----------------------------------------------**********************************************+
 
 URL ANGULAR --------**********************************--------------------------------------------------
 
 http://localhost:4200/ 
 
 se encuentra configurada en en procram.cs de la API para la integracion back - front, si al ejecutar el proyecto de angular el puerto es diferente favor cambiar.
 
------------------------**********************----------------------------------------------------------------------
 
 
 ************** IMPORTANTE*****************-----------------------------*******************************
 
 +++para crear una new order adjunto los datos corespondientes ya que de no llenar todos los campos que son obligatorios 
 +++ no lo permite realizar++++++++++++++++++++++++++++++++++++++++++++++++++++++++++++++++++++++++++++++++
   
Los datos son : 
   
 
  "empId": 9,
  "shipperId": 2,
  "shipName": "Ship to 71-C",
  "shipAddress": "9012 Suffolk Ln.",
  "shipCity": "Boise",
  "orderDate": "2025-03-20T02:42:50.015Z",
  "requiredDate": "2025-03-20T02:42:50.015Z",
  "shippedDate": "2025-03-20T02:42:50.015Z",
  "freight": 79,
  "shipCountry": "Colombia",
  "custId":91 ,
  "orderDetail": {
    "productId": 22,
    "unitPrice": 17,
    "qty": 2,
    "discount": 0.030
  



SalesDatePredictionTest ----------------------------------------


dotnet new xunit -o TuProyectoDePruebas  --- comando para crear el proyecto de pruebas 
dotnet add package xunit  --- agregar paquete xunit



----- para el punto 5 Graficando con D3 -----------------

Se ha creado la carpeta grafico-barras en la cual se tienen tres archivos solo se debe

Cómo ejecutar:

1-Abre index.html en tu navegador web.
2-Ingresa números separados por coma en el input de texto y haz clic en "Update Data".
3-El gráfico de barras se mostrará en el div "chart". 



-------------------------------------------------------------------------------